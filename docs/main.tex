%%%%%%%%%%%%%%%%%%%%%%%%%%%%%%%%%%%%%%%%%
% Wenneker Assignment
% LaTeX Template
% Version 2.0 (12/1/2019)
%
% This template originates from:
% http://www.LaTeXTemplates.com
%
% Authors:
% Vel (vel@LaTeXTemplates.com)
% Frits Wenneker
%
% License:
% CC BY-NC-SA 3.0 (http://creativecommons.org/licenses/by-nc-sa/3.0/)
% 
%%%%%%%%%%%%%%%%%%%%%%%%%%%%%%%%%%%%%%%%%

%----------------------------------------------------------------------------------------
%	PACKAGES AND OTHER DOCUMENT CONFIGURATIONS
%----------------------------------------------------------------------------------------

\documentclass[11pt]{scrartcl} % Font size

\input{structure.tex} % Include the file specifying the document structure and custom commands

%----------------------------------------------------------------------------------------
%	TITLE SECTION
%----------------------------------------------------------------------------------------

\title{	
	\normalfont\normalsize
	\textsc{Instituto Tecnológico de Buenos Aires}\\ % Your university, school and/or department name(s)
	\vspace{25pt} % Whitespace
	\rule{\linewidth}{0.5pt}\\ % Thin top horizontal rule
	\vspace{20pt} % Whitespace
	{\huge Informe Trabajo Práctico Especial}\\ % The assignment title
	\vspace{25pt} % Whitespace
	{\huge Criptografía y Seguridad [72.44] - 2021 1C}\\ % The assignment title
	\vspace{12pt} % Whitespace
	\rule{\linewidth}{2pt}\\ % Thick bottom horizontal rule
	\vspace{12pt} % Whitespace
}

% \author{\LARGE Grupo 16: Luciano Boccardi - Tamara Puig -  Ximena Zuberbuhler} % Your name

\author{
  Boccardi, Luciano (59518)\\
  \texttt{lboccardi@itba.edu.ar}
  \and
  Puig, Tamara (59820)\\
  \texttt{tpuig@itba.edu.ar}
  \and
  Zuberbuhler, Ximena (57287)\\
  \texttt{xzuberbuhler@itba.edu.ar}
}

\date{\normalsize\today} % Today's date (\today) or a custom date

\begin{document}

\maketitle % Print the title

%----------------------------------------------------------------------------------------
%	FIGURE EXAMPLE
%----------------------------------------------------------------------------------------

\newpage

\section{Recuperación del Secreto}

\begin{figure}[h] % [h] forces the figure to be output where it is defined in the code (it suppresses floating)
	\centering
	\fbox{\includegraphics[width=0.5\columnwidth]{big-ben.png}}
	\caption{Imágen recuperada a partir de los archivos dados.}
	\label{secret_image}
\end{figure}

%------------------------------------------------

\subsection{Conclusiones de la ejecución del programa}

El archivo asignado al grupo consistía de 4 imágenes BMP. Se recuperó el secreto con un esquema de 4 imágenes portadoras y el resultado fue la imagen mostrada en la \figurename{\ref{secret_image}}. Habiendo seguido el algoritmo de manera correcta, se puede concluir que la recuperación fue efectiva.

%----------------------------------------------------------------------------------------
%	TEXT EXAMPLE
%----------------------------------------------------------------------------------------

\section{Cuestiones a Analizar}

\subsection{Aspectos relativos al documento}

El paper analizado tiene una estructura muy bien organizada. Primero se realiza una mención del estado del arte, y trabajos previos del mismo tema. Luego, se comienza a describir la idea y la teoría que la respalda. Se explican las etapas de encripción y desencripción con diagramas de flujo, lo cual ayuda a comprender el procedimiento. Por último, se detalla la carga útil y se realizan comparaciones con los métodos detallado al comienzo del informe.

En cuanto a la descripción de los algoritmos, si bien es sumamente clara, se encontraron algunos problemas en las fórmulas. Muchos de los signos '-' no aparecen, e incluso en la fórmula 7 está mal detallada ya que refiere a $s_{0}$ y no a $s_{r-1}$ como debería. No obstante, al ser fórmulas correspondientes a un método de interpolación ampliamente conocido, no hubieron mayores inconvenientes. El resto del detalle es detallado y completo, con un grado de detalle adecuado para poder ser comprendido por personas con una formación correspondiente.

La notación es correcta, no obstante, los índices a los que se refieren suelen entrar en conflicto con los utilizados en las fórmulas matemáticas. Por ejemplo, se utiliza $j$ para denominar el número de bloque, y en la fórmula de interpolación se itera sobre el índice $q$. Quizás hubiera sido mejor optar por definir otros índices para evitar esos conflictos.

Por último, existe un debate cuestionable entre comenzar a indexar con el número 0 o con el número 1. No nos parece relevante discutir sobre cuál es más adecuado, ya que todo depende del lenguaje en el cual se trabaje. A fines de un paper, es correcto porque se mantiene consistente a lo largo del mismo.

%------------------------------------------------

\subsubsection{Optimización de la carga útil}

Cuando en el paper se refiere a la carga útil está hablando de cuánta información se guarda en cada sombra. Si asumimos que el bloque (4 píxeles = 4 bytes) es la menor unidad de información, entonces podemos establecer una comparación.

Suponiendo que todas las shadows son del mismo tamaño que coincide con el tamaño del secreto. Sea $k$ el número de sombras y $L$ el tamaño del secreto, la imagen se divide en $\frac{L}{k}$ conjuntos de $k$ elementos. Por cada conjunto, se guarda 1 elemento en un bloque de $2x2$.

Es fácil notar que si $k = 4$, se utilizan $\frac{L}{k} * 4$ píxeles, que es exactamente el mismo tamaño que la imagen original.

En el caso $k > 4$, se puede definir un factor de carga $f = \frac{4}{k}$ que define qué porcentaje de cada imagen se ocupa guardando el secreto. Para $k = 5, f = 0.8$, y para $k = 6, f = 0.67$.

Puede verse que mientras mayor es $k$, se necesitan más portadoras pero el tamaño requerido de cada una para guardar el secreto es menor.

%------------------------------------------------

\subsubsection{Suppose ``chuck" implies vomiting.}

A woodchuck can ingest 361.92 cm\textsuperscript{3} (22.09 cu in) of wood per day. Assuming immediate expulsion on ingestion with a 5\% retainment rate, a woodchuck could chuck \textbf{343.82 cm\textsuperscript{3}} of wood per day.

%------------------------------------------------

\paragraph{Bonus: suppose there is no woodchuck.}

Fusce varius orci ac magna dapibus porttitor. In tempor leo a neque bibendum sollicitudin. Nulla pretium fermentum nisi, eget sodales magna facilisis eu. Praesent aliquet nulla ut bibendum lacinia. Donec vel mauris vulputate, commodo ligula ut, egestas orci. Suspendisse commodo odio sed hendrerit lobortis. Donec finibus eros erat, vel ornare enim mattis et.

%----------------------------------------------------------------------------------------
%	EQUATION EXAMPLES
%----------------------------------------------------------------------------------------

\section{Interpreting Equations}

\subsection{Identify the author of Equation \ref{eq:bayes} below and briefly describe it in English.}

\begin{align} 
	\label{eq:bayes}
	\begin{split}
		P(A|B) = \frac{P(B|A)P(A)}{P(B)}
	\end{split}					
\end{align}

Lorem ipsum dolor sit amet, consectetur adipiscing elit. Praesent porttitor arcu luctus, imperdiet urna iaculis, mattis eros. Pellentesque iaculis odio vel nisl ullamcorper, nec faucibus ipsum molestie. Sed dictum nisl non aliquet porttitor. Etiam vulputate arcu dignissim, finibus sem et, viverra nisl. Aenean luctus congue massa, ut laoreet metus ornare in. Nunc fermentum nisi imperdiet lectus tincidunt vestibulum at ac elit. Nulla mattis nisl eu malesuada suscipit.

%------------------------------------------------

\subsection{Try to make sense of some more equations.}

\begin{align} 
	\begin{split}
		(x+y)^3 &= (x+y)^2(x+y)\\
		&=(x^2+2xy+y^2)(x+y)\\
		&=(x^3+2x^2y+xy^2) + (x^2y+2xy^2+y^3)\\
		&=x^3+3x^2y+3xy^2+y^3
	\end{split}					
\end{align}

Lorem ipsum dolor sit amet, consectetuer adipiscing elit. 
\begin{align}
	A = 
	\begin{bmatrix}
		A_{11} & A_{21} \\
		A_{21} & A_{22}
	\end{bmatrix}
\end{align}
Aenean commodo ligula eget dolor. Aenean massa. Cum sociis natoque penatibus et magnis dis parturient montes, nascetur ridiculus mus. Donec quam felis, ultricies nec, pellentesque eu, pretium quis, sem.

%----------------------------------------------------------------------------------------
%	LIST EXAMPLES
%----------------------------------------------------------------------------------------

\section{Viewing Lists}

\subsection{Bullet Point List}

\begin{itemize}
	\item First item in a list 
		\begin{itemize}
		\item First item in a list 
			\begin{itemize}
			\item First item in a list 
			\item Second item in a list 
			\end{itemize}
		\item Second item in a list 
		\end{itemize}
	\item Second item in a list 
\end{itemize}

%------------------------------------------------

\subsection{Numbered List}

\begin{enumerate}
	\item First item in a list 
	\item Second item in a list 
	\item Third item in a list
\end{enumerate}

%----------------------------------------------------------------------------------------
%	TABLE EXAMPLE
%----------------------------------------------------------------------------------------

\section{Interpreting a Table}

\begin{table}[h] % [h] forces the table to be output where it is defined in the code (it suppresses floating)
	\centering % Centre the table
	\begin{tabular}{l l l}
		\toprule
		\textit{Per 50g} & \textbf{Pork} & \textbf{Soy} \\
		\midrule
		Energy & 760kJ & 538kJ\\
		Protein & 7.0g & 9.3g\\
		Carbohydrate & 0.0g & 4.9g\\
		Fat & 16.8g & 9.1g\\
		Sodium & 0.4g & 0.4g\\
		Fibre & 0.0g & 1.4g\\
		\bottomrule
	\end{tabular}
	\caption{Sausage nutrition.}
\end{table}

%------------------------------------------------

\subsection{The table above shows the nutritional consistencies of two sausage types. Explain their relative differences given what you know about daily adult nutritional recommendations.}

Lorem ipsum dolor sit amet, consectetur adipiscing elit. Praesent porttitor arcu luctus, imperdiet urna iaculis, mattis eros. Pellentesque iaculis odio vel nisl ullamcorper, nec faucibus ipsum molestie. Sed dictum nisl non aliquet porttitor. Etiam vulputate arcu dignissim, finibus sem et, viverra nisl. Aenean luctus congue massa, ut laoreet metus ornare in. Nunc fermentum nisi imperdiet lectus tincidunt vestibulum at ac elit. Nulla mattis nisl eu malesuada suscipit.

%----------------------------------------------------------------------------------------
%	CODE LISTING EXAMPLE
%----------------------------------------------------------------------------------------

\section{Reading a Code Listing}

\lstinputlisting[
	caption=Luftballons Perl Script., % Caption above the listing
	label=lst:luftballons, % Label for referencing this listing
	language=Perl, % Use Perl functions/syntax highlighting
	frame=single, % Frame around the code listing
	showstringspaces=false, % Don't put marks in string spaces
	numbers=left, % Line numbers on left
	numberstyle=\tiny, % Line numbers styling
	]{luftballons.pl}

%------------------------------------------------

\subsection{How many luftballons will be output by the Listing \ref{lst:luftballons} above?}

Aliquam arcu turpis, ultrices sed luctus ac, vehicula id metus. Morbi eu feugiat velit, et tempus augue. Proin ac mattis tortor. Donec tincidunt, ante rhoncus luctus semper, arcu lorem lobortis justo, nec convallis ante quam quis lectus. Aenean tincidunt sodales massa, et hendrerit tellus mattis ac. Sed non pretium nibh. Donec cursus maximus luctus. Vivamus lobortis eros et massa porta porttitor.

%------------------------------------------------

\subsection{Identify the regular expression in Listing \ref{lst:luftballons} and explain how it relates to the anti-war sentiments found in the rest of the script.}

Fusce varius orci ac magna dapibus porttitor. In tempor leo a neque bibendum sollicitudin. Nulla pretium fermentum nisi, eget sodales magna facilisis eu. Praesent aliquet nulla ut bibendum lacinia. Donec vel mauris vulputate, commodo ligula ut, egestas orci. Suspendisse commodo odio sed hendrerit lobortis. Donec finibus eros erat, vel ornare enim mattis et.

%----------------------------------------------------------------------------------------

\end{document}
